%!TEX encoding = UTF-8 Unicode
%!TEX root = ../easy-bindings.tex


\chapter{Structure globale}



L'entrée du compilateur EasyBindings est un fichier source. Celui-ci peut en inclure d'autres via une déclaration \fcolorbox{white}{green}{\bf include}. D'une manière générale, un fichier source contient une liste de \emph{déclarations} :
\begin{itemize}
  \item déclaration d'une classe d'outlet (\refSectionPage{declarationOutletClasse}) ;
  \item inclusion d'un autre fichier (\refSectionPage{inclusionFichier}) ;
  \item déclaration d'un template de contrôleur (\refChapterPage{declarationTemplateControleur}) ;
  \item déclaration de \emph{préférences} (\refChapterPage{declarationPreferences}) ;
\end{itemize}

L'ordre des déclarations est sans importance.

\sectionLabel{Déclaration d'une classe d'outlet}{declarationOutletClasse}

\fcolorbox{white}{green}{\begin{minipage}{1.\textwidth}\tt
{\bf outletClass} Nom ;
\end{minipage}}

Il n' y a pas de classe prédéclarée. À chaque classe ainsi déclarée, correspond un fichier situé dans \texttt{generation-templates/outlet-classes}. On ne peut donc pas utiliser directement les classes Cocoa, mais via une classe héritière.



\sectionLabel{Inclusion d'un fichier}{inclusionFichier}

\fcolorbox{white}{green}{\begin{minipage}{1.\textwidth}\tt
{\bf include} "fichier.easyBindings" ;
\end{minipage}}

\section{Les types}

