%!TEX encoding = UTF-8 Unicode
%!TEX root = ../easy-bindings.tex


\chapterLabel{Déclaration d'un template de contrôleur}{declarationTemplateControleur}

\fcolorbox{white}{green}{\begin{minipage}{1.\textwidth}\tt
{\bf controllerTemplate} \\
\hspace*{.5cm}NomClasseOutlet \$nomBinding, ... \\
\hspace*{.5cm}: \\
\hspace*{.5cm}NomDeType nomDeModèle, ... \\
\hspace*{.5cm}\{ options \} \\
;
\end{minipage}}

À chaque template correspond un fichier texte. Son nom est obtenu à partir des informations fournies par la déclaration. Par exemple, à la déclaration :
\begin{itemize}
  \item[] \texttt{\small {\bf controllerTemplate} PMTextField \$value : String model \{sendContinuously : Bool\} ;}
\end{itemize}


Correspond le fichier :

\begin{itemize}
  \item[] \texttt{\small PMTextField.value.String.model.txt}
\end{itemize}


Dans ce fichier template, les chaînes \texttt{\$OBJECTCLASS\$} et \texttt{\$MODEL\$} seront remplacées lors de l'instanciation par le nom de la classe contenant l'attribut, et le nom de cet attribut.