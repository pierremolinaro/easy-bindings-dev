%!TEX encoding = UTF-8 Unicode
%!TEX root = ../easy-bindings.tex


\chapterLabel{Propriétés observables}{chapitreProprietesObservables}


\section{Types prédéfinis des propriétés}

\begin{table}[t]
  \centering
  \begin{tabular}{ll}
    \textbf{Type Easy Binding et Swift} & \textbf{Sémantique}\\
    \eb!Bool!   & Valeur \\
    \eb!Int!    & Valeur \\
    \eb!String!   & Valeur \\
    \eb!NSColor!   & Référence \\
    \eb!NSDate!   & Référence \\
    \eb!NSFont!   & Référence \\
    \eb!NSImage!   & Référence \\
  \end{tabular}
  \caption{Types prédéfinis des propriétés}
  \labelTableau{typesProprietes}
  \ligne
\end{table}

Les types prédéfinis des propriétés sont listés dans le \refTableau{typesProprietes}. Les noms de types adoptés en \emph{Easy Bindings} sont les mêmes qu'en Swift.

Il est possible d'ajouter des types de propriétés ayant une sémantique de référence (voir \refSectionPage{ajoutTypePropriete}).
















\section{Propriétés de l'objet courant}


\begin{ebcode}
self.propriete
\end{ebcode}



\section{Propriétés d'un contrôleur de tableau}

Le contrôleur est déclaré dans le même objet.

Accès au tableau des objets filtrés et ordonnés par le contrôleur
\begin{ebcode}
controlleur.sortedArray
\end{ebcode}

Nombre d'objets filtrés et ordonnés par le contrôleur
\begin{ebcode}
controlleur.sortedArray.count
\end{ebcode}

Nombre d'objets sélectionnés par le contrôleur
\begin{ebcode}
controlleur.selectedArray.count
\end{ebcode}

Aucun objet sélectionné par le contrôleur
\begin{ebcode}
controlleur.selection.empty
\end{ebcode}

Plusieurs objets sélectionnés par le contrôleur
\begin{ebcode}
controlleur.selection.multiple
\end{ebcode}

Un seul objet sélectionné par le contrôleur
\begin{ebcode}
controlleur.selection.unique
\end{ebcode}


\section{Graphe d'héritage des classes des simples propriétés}

Les propriétés ont des valeurs atomiques. Elles ont :
\begin{itemize}
\item soit une sémantique de \emph{valeur}, comme \eb+Int+, \eb+String+, ... ; 
\item soit une sémantique de \emph{référence}, comme \eb+NSFont+, ... 
\end{itemize}

\subsection{Propriétés à sémantique de valeur}

Les propriétés à sémantique de valeur prédéfinies sont : \eb+Int+, \eb+Double+, \eb+String+ et \eb+Bool+. L'utilisateur peut en définir de nouvelles qui sont des énumérations ou des structures.

\begin{figure}[t]
  \centering
  \small
  \texttt{
    \begin{tikzpicture}
      \node[rectangle,draw=red] (absProp) {PMAbstractProperty} ;
      \node[rectangle,draw=red, below=of absProp] (readOnly) {EBReadOnlyValueProperty <T>} ;
      \node[rectangle,draw=red, below=of readOnly] (readWrite){EBReadWriteValueProperty <T>} ;
      \node[rectangle,draw, below= 2cm of readWrite] (proxy) {EBPropertyValueProxy <T : ValuePropertyProtocol>} ;
      \node[rectangle,draw, below right=of readWrite] (stored) {EBStoredValueProperty <T : ValuePropertyProtocol>} ;
      \node[rectangle,draw, right=of readOnly] (transient){EBTransientValueProperty <T>} ;
      \draw [-stealth, thick] (readOnly) -- (absProp) ;
      \draw [-stealth, thick] (transient) -- (readOnly) ;
      \draw [-stealth, thick] (readWrite) -- (readOnly) ;
      \draw [-stealth, thick] (stored) -- (readWrite) ;
      \draw [-stealth, thick] (proxy) -- (readWrite) ;
    \end{tikzpicture}
  }
  \caption{Classes génériques des propriétés à sémantique de valeur}
  \labelFigure{classesGeneriquesPropValeur}
  \ligne
\end{figure}

\subsection{Propriétés à sémantique de référence}

Les propriétés à sémantique de valeur prédéfinies sont : \eb+NSColor+, \eb+NSData+, \eb+NSFont+ et \eb+NSDate+. L'utilisateur peut en définir de nouvelles.

Définir de nouvelles classes de propriété \emph{transient} à valeur de référence se fait par la déclaration :
\begin{ebcode}
transient property class MyClass ;
\end{ebcode}

Seules les classes \texttt{EBReadOnlyClassProperty<MyClass>} et \texttt{EBTransientClassProperty<MyClass>} sont définies, aussi il n'est pas nécessaire que la classe \texttt{MyClass} implémente le protocole \texttt{ClassPropertyProtocol}. 

\begin{figure}[t]
  \centering
  \small
  \texttt{
    \begin{tikzpicture}
      \node[rectangle,draw=red] (absProp) {PMAbstractProperty} ;
      \node[rectangle,draw=red, below=of absProp] (readOnly) {EBReadOnlyClassProperty <T>} ;
      \node[rectangle,draw=red, below=of readOnly] (readWrite){EBReadWriteClassProperty <T>} ;
      \node[rectangle,draw, below= 1.5cm of readWrite] (proxy) {EBPropertyClassProxy <T : ClassPropertyProtocol>} ;
      \node[rectangle,draw, below= 2cm of transient] (stored) {EBStoredClassProperty <T : ClassPropertyProtocol>} ;
      \node[rectangle,draw, right=of readOnly] (transient){EBTransientClassProperty <T>} ;
      \draw [-stealth, thick] (readOnly) -- (absProp) ;
      \draw [-stealth, thick] (transient) -- (readOnly) ;
      \draw [-stealth, thick] (readWrite) -- (readOnly) ;
      \draw [-stealth, thick] (stored) -- (readWrite) ;
      \draw [-stealth, thick] (proxy) -- (readWrite) ;
    \end{tikzpicture}
  }
  \caption{Classes génériques des propriétés à sémantique de référence}
  \labelFigure{classesGeneriquesPropReference}
  \ligne
\end{figure}



\begin{figure}[t]
  \newcommand\classe[2] {
   {\nodepart{one}\textbf{#1}\nodepart {two}\small #2}
  }
  \centering
  \texttt{
    \begin{tikzpicture}
      \tikzset{
         bnode2/.style = {
            rectangle, draw,
            rectangle split,
            rectangle split parts=2
         },
         bnode3/.style = {
            rectangle, draw,
            rectangle split,
            rectangle split parts=3
         }
      }
      \node[bnode3] (absProp) {
        \nodepart{one}\bf PMAbstractProperty\nodepart{two}\footnotesize addObserver \nodepart{three}\footnotesize removeObserver
      } ;
      \node[below=of absProp, bnode2] (readOnly) {
        \nodepart{one}\bf PMReadOnlyProperty\_String
        \nodepart {two}\footnotesize var prop : EBProperty <String> \{ get \}
      } ;
      \node[bnode3, below=of readOnly] (readWrite){
        \nodepart{one}\bf PMReadWriteProperty\_String
        \nodepart{two}\footnotesize func setProp (inValue : String)
        \nodepart{three}\footnotesize func validateAndSetProp (...) -> ...
      } ;
      \node[rectangle,draw, below=of readWrite] (proxy) {\bf PMPropertyProxy\_String} ;
      \node[rectangle,draw, below right=of readWrite] (stored) {\bf PMStoredProperty\_String} ;
      \node[rectangle,draw, right=of readOnly] (transient){\bf PMTransientProperty\_String} ;
      \draw [-stealth, thick] (readOnly) -- (absProp) ;
      \draw [-stealth, thick] (transient) -- (readOnly) ;
      \draw [-stealth, thick] (readWrite) -- (readOnly) ;
      \draw [-stealth, thick] (stored) -- (readWrite) ;
      \draw [-stealth, thick] (proxy) -- (readWrite) ;
    \end{tikzpicture}
  }
  \caption{Graphe d'héritage de la propriété \texttt{String}}
  \labelFigure{figureRepertoireLOGOapresCompGALGAS}
  \ligne
\end{figure}








\sectionLabel{Ajouter des types de propriété}{ajoutTypePropriete}


Il est possible de définir des nouveaux types qui seront utilisables pour des propriétés, et pour lesquels on pourra définir des bindings. Il faut faire une distinction entre le type qui est déclaré en \emph{Easy Binding}, et ceux utilisés dans le code swift engendré. Par exemple, si on désire utiliser le type \texttt{NSData}, on écrira :

\begin{ebcode}
property class NSData (empty) ;
\end{ebcode}

Cette déclaration nomme le type déclaré \eb!NSData! et, entre parenthèses, l'initialiseur \eb!empty!\footnote{Plusieurs initialiseurs peuvent être déclarés, il suffit de les séparer par des virgules.}.

Le code Swift engendré ne déclare pas le type \texttt{NSData}, il suppose qu'il est utilisable dans le projet, en l'occurence il est défini dans l'\emph{Application Kit}. 

Par contre, la compilation \emph{Easy Binding} engendre les classes \texttt{EBReadOnlyClassProperty\_NSData}, \texttt{PMPropertyProxy\_NSData}, \texttt{PMStoredProperty\_NSData}, \texttt{PMTransientProperty\_NSData}.

Quand on déclare une propriété en \emph{Easy Binding}, il faut préciser sa valeur initiale. Celle-ci est mentionnée après le mot réservé \eb!default!. Par exemple :

\begin{ebcode}
property class NSData (empty) ;

prefs {
  property NSData mData default empty ;
}
\end{ebcode}

Si on compile le code engendré avec Xcode, on voit apparaître une erreur sur la ligne :
\begin{lstlisting}[language=swift]
  var mData = EBStoredProperty_NSData (NSData.empty ())
\end{lstlisting}

En effet, \texttt{empty} n'est pas un initialiseur valide de \texttt{NSData} : il faudrait écrire \texttt{NSData ()} pour instancier un objet de type \texttt{NSData}.

Pour corriger l'erreur, if suffit d'écrire une fonction statique \texttt{empty} dans une extension de \texttt{NSData} :
\begin{lstlisting}[language=swift]
extension NSData {
  static func empty () -> NSData { return NSData () }
}
\end{lstlisting}

Cette extension peut être placée n'importe où dans le projet, la visibilité des extension étant globale. Il faut simplement ne pas la placer dans un fichier qui sera ré-engendré lors d'une recompilation ultérieure : la modification sera perdue. On peut simplement créer un nouveau fichier qui contient cette extension, et l'ajouter au projet.










\sectionLabel{Ajouter des types de propriété transitoire}{ajoutTypeProprieteTransitoire}


Il est possible de définir des nouveaux types qui seront utilisables pour des propriétés transitoires, c'est-à-dire les propriétés déclarées \eb!transient!. Par exemple, le type \eb!NSImage! est prédéclaré ainsi en \emph{Easy Bindings} :
\begin{ebcode}
transient property class NSImage ;
\end{ebcode}

Le code Swift engendré ne déclare pas le type \texttt{NSImage}, il suppose qu'il est utilisable dans le projet, en l'occurence il est défini dans l'\emph{Application Kit}. 

Par contre, la compilation \emph{Easy Binding} engendre les classes \texttt{EBReadOnlyClassProperty\_NSImage} et \texttt{PMTransientProperty\_NSImage}.



